% This is LLNCS.DEM the demonstration file of
% the LaTeX macro package from Springer-Verlag
% for Lecture Notes in Computer Science,
% version 2.3 for LaTeX2e
%
\documentclass{llncs}
%
\usepackage{makeidx}  % allows for indexgeneration
\usepackage{graphicx}
\usepackage{amssymb}
\usepackage{listings}
\usepackage{tabularx}
\usepackage{booktabs}
\usepackage{subscript}
\lstset{breaklines=true, basicstyle=\scriptsize\ttfamily}

%
\begin{document}
%
\frontmatter          % for the preliminaries
%
\pagestyle{headings}  % switches on printing of running heads
%\addtocmark{Hamiltonian Mechanics} % additional mark in the TOC
%
%
\mainmatter              % start of the contributions
%
\title{Keeping scientific materials alive and reusable by supporting collaboration}

%
\titlerunning{Keeping scientific materials alive and reusable by supporting collaboration}  % abbreviated title (for running head)
%                                     also used for the TOC unless
%                                     \toctitle is used
%
\author{Aleix Garrido\inst{1} \and Jose Manuel Gomez-Perez\inst{1} \and Boris Villazon-Terrazas\inst{1}}
%
\authorrunning{Garrido et al.}   % abbreviated author list (for running head)
%
%%%% list of authors for the TOC (use if author list has to be modified)
%\tocauthor{Panos Alexopoulos, Manolis Wallace}
%
\institute{
iSOCO, Intelligent Software Components S.A., Av. del
Partenon, 16-18, 1-7, 28042, Madrid, Spain,\\
\email{\{agarrido,jmgomez,bvillazon\}@isoco.com}}


\maketitle              % typeset the title of the contribution

%**************************************************************
%**************************************************************
%**************************************************************


\begin{abstract}
Data-intensive scientific knowledge faces many issues such as the decrease of quality of materials and the difficulties on finding relevant information. Those aspects need to be addressed in order to enhance the future of scientific development. This article presents a set of tools focused on reusability of scientific information based on two different aspects: collaboration and search. We present a visual metaphor called Collaboration Spheres which allow search through exploration of scientific social networks based on a combination of people and information objects. This search paradigm is built around the use of customizable contexts that act as intuitive examples conducted by the users. This tool eases collaboration, expert finding and access to relevant information.
\end{abstract}

\section{Introduction and motivation}\label{sec:Introduction}
Automatic review assignment is very beneficial for many people such as conference organizers, journal editors, and grant administrators. Currently, within the APA Vivo Platform \cite{harren_2012} this review assignment task is done manually. 

The APA Vivo Platform is ...

The Collaboration Spheres \cite{} are a well-know system within the Workflow Scientific community. CSs focus on the combination of recommendation technologies and exploratory search user interfaces for a powerful and simplified access to repositories in scientific communities. CSs aim at providing a mechanism to explore, share and reuse information objects and user expertise based on the exploitation of semantic descriptions, relations, and similarities between information objects and users in order to provide advanced search functionalities.

In this paper we present an application, Expert Finder, aiming to help journal editors, conference and workshop chairs on identifying suitable reviewers for a particular paper, by providing a nice visualisation of related authors and papers. To this end, the application relies on an innovative social network visualisation, namely Collaboration Spheres. Basically, we have adapted the already existing CSs for building the Expert Finder system.



 


%\section{}\label{sec:RelatedWork}
%Reuse of scientific work, including search and exploration of relevant scientific experiments in scientific social networks (Collaboration Spheres). Wf4Ever 


\section{Expert finder}\label{sec:Motivation}
Reviewer finder, illustrated by the APA prototype 




\section{Dataset Information}\label{sec:DataInfo}
In this section we provide information related with the vocabularies that are  used in the dataset together with additional stats about the amount of information managed by the application.\\
Vocabularies:

\begin{tabular}{ l l l }
  1 & VIVO Ontology & http://vivoweb.org/ontology/core \\
  2 & BIBO Ontology & http://purl.org/ontology/bibo \\
  3 & FAO Geopolotical  & http://aims.fao.org/aos/geopolitical.owl \\
  4 & Vitro Application Ontology  & http://vitro.mannlib.cornell.edu/ns/vitro/0.7 \\
  5 & FOAF  & http://xmlns.com/foaf/0.1 \\
  6 & SKOS  & http://www.w3.org/2004/02/skos/core \\
  7 & iSOCO vocab & http://vocab.isoco.net \\ 
  8 & Event Ontology & http://purl.org/NET/c4dm/event.owl \\       
\end{tabular}

Some stats:

\begin{tabular}{ l r  }
  Number of triples & 2605589 \\
  Number of links &   123606  \\
  Number of authors & 12090   \\
  Number of papers & 17386 \\
  Number of topics & 465706 \\
\end{tabular}

%\section{Evaluation}\label{sec:Evaluation}
%Validation and evaluation of the experiment results and materials (Completeness, Stability, Reliability). 



\section{Conclusions and Future Work}\label{sec:Conclusions}
This is the worst conclusions ever ...

In this paper we have described a system for finding reviewers for scholarly publications, developed in the context of a joint project with APA. Future work includes the evaluation of the tool in real case scenario with the participation of experts. 


%*****************************************************************************************************************



\section*{Acknowledgments}

This work was supported by SmartContent project.
%and PARLANCE (www.parlanceproject.eu, grant number 287615).



% ---- Bibliography ----
%

\bibliography{bibliography}
\bibliographystyle{plain}



\clearpage
\addtocmark[2]{Author Index} % additional numbered TOC entry
\renewcommand{\indexname}{Author Index}
\printindex \clearpage
\addtocmark[2]{Subject Index} % additional numbered TOC entry
\markboth{Subject Index}{Subject Index}
\renewcommand{\indexname}{Subject Index}
%\input{subjidx.ind}
\end{document}
