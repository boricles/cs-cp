% This is LLNCS.DEM the demonstration file of
% the LaTeX macro package from Springer-Verlag
% for Lecture Notes in Computer Science,
% version 2.3 for LaTeX2e
%
\documentclass{llncs}
%
\usepackage{makeidx}  % allows for indexgeneration
\usepackage{graphicx}
\usepackage{amssymb}
\usepackage{listings}
\usepackage{tabularx}
\usepackage{booktabs}
\usepackage{subscript}
\lstset{breaklines=true, basicstyle=\scriptsize\ttfamily}

%
\begin{document}
%
\frontmatter          % for the preliminaries
%
\pagestyle{headings}  % switches on printing of running heads
%\addtocmark{Hamiltonian Mechanics} % additional mark in the TOC
%
%
\mainmatter              % start of the contributions
%
\title{Who Reviews this Article? Collaboration Spheres in support of Scholarly Publication}

%
\titlerunning{Who Reviews this Article? Collaboration Spheres in support of Scholarly Publication}  % abbreviated title (for running head)
%                                     also used for the TOC unless
%                                     \toctitle is used
%
\author{Aleix Garrido\inst{1} \and Jose Manuel Gomez-Perez\inst{1} \and Boris Villazon-Terrazas\inst{1}}
%
\authorrunning{Garrido et al.}   % abbreviated author list (for running head)
%
%%%% list of authors for the TOC (use if author list has to be modified)
%\tocauthor{Panos Alexopoulos, Manolis Wallace}
%
\institute{
iSOCO, Intelligent Software Components S.A., Av. del
Partenon, 16-18, 1-7, 28042, Madrid, Spain,\\
\email{\{agarrido,jmgomez,bvillazon\}@isoco.com}}


\maketitle              % typeset the title of the contribution

%**************************************************************
%**************************************************************
%**************************************************************


\begin{abstract}
Science is incremental and cyclic. New breakthroughs are built on top of previous results which will themselves be the basis for future developments once validated by peer scientists and accepted for publication in scholarly communications. However, the process of identifying suitable reviewers for a particular article can be cumbersome and costly. Not only do journal editors need to have a thorough, personal knowledge of their journals and the members of the community who can be suitable reviewers but also they must have an eye for spotting potential conflicts of interest that may raise biased reviews. Thus, means are required that assist editors in order to identify suitable reviewers amongst members of scientific communities. In this paper we present an expert finding system based on the application of the Collaboration Spheres search-by-example visual interface, to the case of the American Psychologic Association, a publishing body with over 70 journals, 54 interest groups, and 134,000 members, including researchers, educators, clinicians, consultants and students. We show how we enriched APA's previously existing RDF data model and produced an extended dataset which explicitly materializes the relationships between authors, articles, and their organizations. Based on this dataset we describe a service layer implemented as a set of SPARQL queries which feeds the Collaboration Spheres interface and enables the required expert finding capabilities. 
\end{abstract}

\section{Introduction}\label{sec:Introduction}
Automatic review assignment is very beneficial for many people such as conference organizers, journal editors, and grant administrators. Currently, within the APA Vivo Platform \cite{harren_2012} this review assignment task is done manually. 

The APA Vivo Platform is ...

The Collaboration Spheres \cite{} are a well-know system within the Workflow Scientific community. CSs focus on the combination of recommendation technologies and exploratory search user interfaces for a powerful and simplified access to repositories in scientific communities. CSs aim at providing a mechanism to explore, share and reuse information objects and user expertise based on the exploitation of semantic descriptions, relations, and similarities between information objects and users in order to provide advanced search functionalities.

In this paper we present an application, Expert Finder, aiming to help journal editors, conference and workshop chairs on identifying suitable reviewers for a particular paper, by providing a nice visualisation of related authors and papers. To this end, the application relies on an innovative social network visualisation, namely Collaboration Spheres. Basically, we have adapted the already existing CSs for building the Expert Finder system.



 


%\section{Reuse of scientific work}\label{sec:RelatedWork}
%Reuse of scientific work, including search and exploration of relevant scientific experiments in scientific social networks (Collaboration Spheres). Wf4Ever 




\section{Expert finder}\label{sec:Motivation}
Reviewer finder, illustrated by the APA prototype 




%\section{Dataset description}



%\section{Evaluation}\label{sec:Evaluation}
%Validation and evaluation of the experiment results and materials (Completeness, Stability, Reliability). 



\section{Conclusions and Future Work}\label{sec:Conclusions}
This is the worst conclusions ever ...

In this paper we have described a system for finding reviewers for scholarly publications, developed in the context of a joint project with APA. Future work includes the evaluation of the tool in real case scenario with the participation of experts. 


%*****************************************************************************************************************



\section*{Acknowledgments}

This work was supported by SmartContent project.
%and PARLANCE (www.parlanceproject.eu, grant number 287615).



% ---- Bibliography ----
%

\bibliography{bibliography}
\bibliographystyle{plain}



\clearpage
\addtocmark[2]{Author Index} % additional numbered TOC entry
\renewcommand{\indexname}{Author Index}
\printindex \clearpage
\addtocmark[2]{Subject Index} % additional numbered TOC entry
\markboth{Subject Index}{Subject Index}
\renewcommand{\indexname}{Subject Index}
\input{subjidx.ind} 
\end{document}
