%introducción
In this scenario the editor in chief of the Journal of Personality and Social Psychology\footnote{\url{http://www.apa.org/pubs/journals/psp/}} is trying to select reviewers for a submission they have just received. The first step will be to open the collaboration spheres for that Article: 
\begin{itemize}
  \item The selected Article is called "Attachment change processes in the early years of marriage" \footnote{\url{http://vivo.apa.org/individual/n61963}}.
  \item Then we can launch the Collaboration Spheres for the Article \footnote{\url{http://apaproject.isoco.com/index.html?id=http://vivo.apa.org/individual/n61963}}.
\end{itemize}
As we can see, the Article the tag cloud shows the most relevant topics for the Article placed in the center of the Collaboration Spheres and we get a preliminary set of recommended reviewers for it.\\
Since we want to define a bit better the topics for our context, i.e. the special issue, and, in order to get better recommendations, we will add an article of our interest (can be found at the “Articles by Topic section), which is very related to the topics of the special issue. We select the following article in the “Articles by topic” panel on the right and drag and drop it into the blue circle of the context of interest:
\begin{itemize}
  \item The added article is "Adult attachment and the transition to parenthood" \footnote{\url{http://vivo.apa.org/individual/n5724}}.
\end{itemize}
Now the editor in chief will include some of the members of the Editorial Board (Interpersonal Relations and Group Processes Section\footnote{\url{http://www.apa.org/pubs/journals/psp/edboard-irgp.aspx}}), who will serve as exemplars of the knowledge required to properly evaluate the submissions which are received. Suitable reviewers will therefore be knowledgeable of (part of) those topics. This way we increase the alignment between the related topics in the expertise of our editors and the recommended experts for reviewing the article. In this case, the editor in chief adds:
\begin{itemize}
  \item Consulting Editors:
  \begin{itemize}
    \item Bolger, Niall
    \item Algoe, Sara B.
  \end{itemize}
  \item Associate Editors:
  \begin{itemize}
      \item Finkel, Eli J.
      \item Gable, Shelly L.
  \end{itemize}
\end{itemize}
We can see at every addition how the main topics change in the tag cloud and how the recommendations are adjusted to the context. When the editor in chief adds the last of the associate editors, the relevance of the different related topics seems to be uniformly distributed. The editor in chief then may want to reconsider the composition of the board and replace some of the editors (e.g. Gable with another expert with knowledge about other topics more, which the editor in chief finds especially relevant for the special issue).  In this case, the editor in chief selects an author who brings in expertise related to topics like alcohol abuse:
\begin{itemize}
  \item W. Zywiak
\end{itemize}
In addition to the summary that we provide for each author and article, the system can also launch a Twitter search that covers his/her/its intersection with the most relevant topics of the context of interest.
